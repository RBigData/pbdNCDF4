
\section[Collective I/O of \pkg{pbdNCDF4}]{Collective I/O of \pkg{pbdNCDF4}}
\label{sec:collective}
\addcontentsline{toc}{section}{\thesection. Collective I/O of \pkg{pbdNCDF4}}

There are mainly three parallel functions added to \pkg{pbdNCDF4} over \pkg{ncdf4}, which
enable collective I/O.  Namely, we include the \proglang{R} functions \code{nc_create_par()},
\code{nc_open_par()}, and \code{nc_var_par_access()}.
\begin{itemize}
\item \code{nc_create_par()} is similar to \code{nc_create()} in \pkg{ncdf4},
but creates a NetCDF4 file with parallel format.

\item \code{nc_open_par()} is similar to \code{nc_open()} in \pkg{ncdf4},
but is specifically for files in the parallel format.

\item \code{nc_var_par_access()} is used to tell the NetCDF4 library to use
collective read or write for the given variable.
\end{itemize}

By default, \code{nc_var_par_access()} sets \code{collective=TRUE}, which
turns on collective read and write for the given variable.
If \code{nc_var_par_access()} is not called or \code{collective=FALSE} is set,
then the independent method will be used.

If parallel versions of HDF5 and NetCDF4 libraries are not available
or \code{-}\code{--enable-parallel} is not set when compiling \pkg{pbdNCDF4},
then user needs to use the original \pkg{ncdf4} functions such as
\code{nc_create} and \code{nc_open}.
For reading in parallel, it will be a little bit slower than collective reading.
However, the serial version should not be used for parallel writing
unless manual synchronization is used.
%However, for writing, the serial version need more \code{barrier} and
%manually synchronization which means much slower than collective writing.
%then these three functions behave as their serial counterparts.  In this case, the 
%\code{nc_create_par()} function will call \code{nc_create()} on processor 0 and
%call \code{nc_open()} on other processors (if any) with \code{write=TRUE}.  Likewise,
%\code{nc_open_par()} calls \code{nc_open()} on all processors, while
%\code{nc_var_par_access()} will do nothing on all processors.

%This behavior allows \pkg{pbdNCDF4} to be fully backwards compatible with \pkg{ncdf4}.  For instance, we can run the serial example as in the
We can run the serial example as in the
Section~\ref{sec:quick_example} next without parallel read and write
capabilities.
Suppose serial HDF5 and NetCDF4
are compiled and \pkg{pbdNCDF4} is installed without the enable parallel flag
\code{-}\code{--enable-parallel}.
\begin{Command}
### Under command mode, run the demo with 2 processors by
### (Use Rscript.exe for windows system)

mpiexec -np 2 Rscript -e "demo(ncwrite_ser,'pbdNCDF4',ask=F,echo=F)"
mpiexec -np 2 Rscript -e "demo(ncread_ser,'pbdNCDF4',ask=F,echo=F)"
\end{Command}
The first \code{demo()} uses two processors and
independently writes to the file
\code{test_ser.nc} in order to generate the same matrix,
while the second \code{demo} reads back in and print the matrix in
serial. The outputs are exactly the same as the parallel version.
Note that the first demo \code{ncwrite_ser} provides an example of
manual synchronization, see the source code for details.

