
\section[Windows System]{Windows System}
\label{sec:windows_system}
\addcontentsline{toc}{section}{\thesection. Windows System}

Windows system has serial HDF5 and NetCDF4 libraries available in
both 32- and 64-bit systems. The pre-built (\pkg{netCDF-C}) libraries can be
downloaded from
\url{http://www.unidata.ucar.edu/software/netcdf/docs/winbin.html}.
For example, the stable release (\pkg{netCDF-C 4.3.2}) has
\begin{itemize}
\item 32-bit library at \\
\url{http://www.unidata.ucar.edu/netcdf/win_netcdf/netCDF4.3.2-NC4-32.exe}, and
\item 64 bit library at \\
\url{http://www.unidata.ucar.edu/netcdf/win_netcdf/netCDF4.3.2-NC4-64.exe}.
\end{itemize}
Note that both pre-built libraries contain HDF5 and NetCDF4 serial versions
and can be directly linked and loaded with \pkg{pbdNCDF4} inside \proglang{R}.

All required \code{*.dll} files are via \code{dyn.load()} in
\code{.onLoad()} at run time, and via \code{dyn.unload()} in
\code{.onUnload()} when quiting the package.

With a few click on both \code{*.exe} files,
we suggest users to install at the system directory
\begin{itemize}
\item
\code{C:/Program Files (x86)/netCDF 4.3.2/} for the 32-bit library, and
\item
\code{C:/Program Files/netCDF 4.3.2/} for the 64-bit library.
\end{itemize}
Note that the default for 32-bit may be the same as 64-bit, so do the change
manually since they can not be merged in general.


\subsection[Install from Binary]{Install from Binary}
\label{sec:install_from_binary}
\addcontentsline{toc}{subsection}{\thesubsection. Install from Binary}

The binary packages of \pkg{pbdNCDF4} are available on the website:
``Programming with Big Data in R'' at
\url{http://r-pbd.org/}.
The binary can be installed by
\begin{Command}
R CMD INSTALL pbdNCDF4_0.1-3.zip
\end{Command}

As in Unix systems,
one can start quickly with \pkg{pbdNCDF4} by learning from the
following demos. There are two basic examples in serial.
\begin{Command}
demo(ncwrite_ser,'pbdNCDF4',ask=F,echo=F)
demo(ncread_ser,'pbdNCDF4',ask=F,echo=F)
\end{Command}


\subsection[Build from Source]{Build from Source}
\label{sec:building_from_source}
\addcontentsline{toc}{subsection}{\thesubsection. Build from Source}

Installation of \pkg{pbdNCDF4} in windows system requires users to set
environment variables \code{NETCDF4_ROOT_32} and
\code{NETCDF4_ROOT_64} for both pre-built libraries.
By default, we suggest
\begin{Command}
### Under command mode, or save in a batch file.
SET NETCDF4_ROOT_32=C:\Program Files (x86)\netCDF 4.3.2\
SET NETCDF4_ROOT_64=C:\Program Files\netCDF 4.3.2\
SET NETCDF4_ROOT=%NETCDF4_ROOT_64%

SET R_HOME=C:\Program Files\R\R-3.0.1\
SET RTOOLS=C:\Rtools\bin\
SET MINGW=C:\Rtools\gcc-4.6.3\bin\

SET PATH=%R_HOME%;%R_HOME%bin\;%RTOOLS%;%MINGW%;%PATH%
SET PATH=%NETCDF4_ROOT%bin\;%PATH%
\end{Command}
Here, we use 64-bit system for testing and set the \code{bin} directory
to \code{PATH}, then
some utilities such as \code{ncdump.exe} can be tested and used in
\pkg{pbdNCDF4}.

With a correct \code{PATH}, one can use the \proglang{R} commands
to install/build the \pkg{pbdNCDF4}:
\begin{Command}
### Under command mode, build and install the binary.
tar zxvf pbdNCDF4_0.1-3.tar.gz
R CMD INSTALL --build pbdNCDF4
R CMD INSTALL pbdNCDF4_0.1-3.zip
\end{Command}


\subsection[Detail Steps]{Detail Steps}
\label{sec:detail_steps}
\addcontentsline{toc}{subsection}{\thesubsection. Detail Steps}

The steps of building \pkg{pbdNCDF4} in Windwos are described next:
\begin{enumerate}
\item
At configure time, I take \code{NETCDF4_ROOT_32} and \code{NETCDF4_ROOT_64}
from environment variables, then set \code{NETCDF4_ROOT} according to
architecture.
\item
Before compile time, I check if the directories exists. If not, then I will
set \code{NETCDF4_LINKED = FALSE} and feed
a fake \code{a.c} to build a fake library in order to pass \code{R CMD check}.
Note that this will eventually cause crashes at run time, so
please check out \pkg{pbdNCDF4} binary from thirteen-01 or dropbox.
\item
At compile time, I save
\code{NETCDF4_ROOT_32}, \code{NETCDF4_ROOT_64}, and \code{NETCDF4_ROOT}
in \code{pbdNCDF4/etc/i386/Makeconf} and \code{pbdNCDF4/etc/x64/Makeconf}.
\item
At installation time, both files will be copied to
\begin{center}
\code{C:/Program Files/R/R-3.0.1/library/pbdNCDF4/...},
\end{center}
with built \code{*.dll} and others.
Then, \code{pbdNCDF4/R/windows/zzz.R} will be copied to
\code{pbdNCDF4/R/zzz.R} to finish compiling \proglang{R} codes.
\item
At load time, \code{.onLoad()} of \code{pbdNCDF4/R/windows/zzz.R} will
obtain \code{NETCDF4_ROOT_32}, \code{NETCDF4_ROOT_64}, and \code{NETCDFR_ROOT}
from system environment variables via \code{Sys.getenv()} first.
\item
If environment variables were unavailable, \code{.onLoad()} will
obtain them from corresponding \code{Makeconf}.
\item
At run time, if none of \code{*.dll} were available or
\code{NETCDF4_LINKED = FALSE}, then I would cast warnings.
Note that any \pkg{pbdNCDF4} function evokes \code{.C()} function may
cause crashes in this case.
\end{enumerate}

In order to load \pkg{pbdNCDF4} correctly in windows, there are several
ways can solve the problem:
\begin{itemize}
\item
There are at least two ways to avoid dynamic file loading problems
if any. Either change the \code{Makeconf} files to the personal/new
installation of \pkg{netCDF-C} or change environment variables.
\item
There are also at least two ways to set environment variables in windows,
either from system ``Control Panel'' or use batch file.
\item
Further, inside \proglang{R}, \code{Sys.setenv()} can be used to
set environment variables in run time.
\end{itemize}
