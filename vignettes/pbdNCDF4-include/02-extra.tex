
\section{Non-Standard System Installations}
\label{sec:useful_info}
\addcontentsline{toc}{section}{\thesection. Non-Standard System Installations}

In this section we discuss a non-default method of installation for \pkg{pbdNCDF4}, as well as compiling the HDF5 and NetCDF4 libraries in parallel.

\subsection[Special Path for NetCDF4]{Special Path for NetCDF4}
\label{sec:special_path}
\addcontentsline{toc}{subsection}{\thesubsection. Special Path for NetCDF4}

As the with the \pkg{ncdf4} package, \pkg{pbdNCDF4} allows special
configuration for \code{nc-config} via the flag \code{-}\code{--with-nc_config}.  For example, we might issue the command:
\begin{Command}
R CMD INSTALL pbdNCDF4 \
  --configure-args="--with-nc-config=/usr/local/netcdf4/bin"
\end{Command}
which specifies that \code{nc-config} is in the directory \code{/usr/local/netcdf4/bin}.


\subsection[Parallel HDF5 and NetCDF4]{Parallel HDF5 and NetCDF4}
\label{sec:parallel_libraries}
\addcontentsline{toc}{subsection}{\thesubsection. Parallel HDF5 and NetCDF4}

Here we assume that the MPI headers are in \code{/usr/include/mpi} and the MPI libraries are in
\code{/usr/lib}. 

We can install HDF5 with parallel I/O via:
\begin{CodeOutput}[title=Compile parallel HDF5]
./configure \
    --prefix=/usr/local/hdf5 \
    --enable-parallel \
    --enable-shared \
    CC="mpicc -g" \
    CFLAGS="-fPIC -I/usr/include/mpi" \
    CPPFLAGS="-fPIC -I/usr/include/mpi" \
    LDFLAGS="-L/usr/lib -lmpi"
make
make install
\end{CodeOutput}

For parallel NetCDF4, suppose that the parallel HDF5 library is installed in \code{/usr/local/hdf5}. Then we can install
NetCDF4 with parallel I/O via:
\begin{Command}[title=Compile parallel NetCDF4]
./configure \
    --prefix=/usr/local/netcdf4 \
    --enable-netcdf4 \
    --enable-shared \
    CC="mpicc -g" \
    CFLAGS="-fPIC -I/usr/include/mpi -I/usr/local/hdf5/include" \
    CPPFLAGS="-fPIC -I/usr/include/mpi -I/usr/local/hdf5/include" \
    LDFLAGS="-L/usr/lib -lmpi -L/usr/local/hdf5/lib -lhdf5"
make
make install
\end{Command}

